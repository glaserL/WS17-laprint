% !TEX root = ../main.tex
\section{Conclusion} % (fold)
\label{sec:conclusion}

In this paper we assessed the recent research on fingerprinting and tested some of the techniques with the help a web service. As discriminating as the technique has been shown to be, there must be cases where it is not. Especially institutions that issue a large number of identically configured devices that are running the exact same OS configurations and possibly are updated centrally altogether, will create one or a small number of anonymity sets. Thus, fingerprinting likely will not replace previous ways of identifying users like cookies or log-ins. Despite these limitations, it can prove to be a valuable addition to other identification techniques. In our use case, fingerprinting could potentially be a initial identification technique that can personalize content for a new visiting user until he decides to \textit{upgrade} to a standard user account. Also, when improving recommendation systems, the number of associated visitors of some educational material might be small or diverse enough for fingerprinting to be sufficient to uniquely identify the set of visitors. 

% section introduction (end)